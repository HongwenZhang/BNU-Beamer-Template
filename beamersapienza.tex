\documentclass{beamer}
\usepackage{amsfonts,amsmath,oldgerm}
\usetheme{sintef}
\usepackage{xeCJK}

\newcommand{\testcolor}[1]{\colorbox{#1}{\textcolor{#1}{test}}~\texttt{#1}}

\usefonttheme[onlymath]{serif}

\titlebackground*{assets/bnu_background}

\newcommand{\hrefcol}[2]{\textcolor{cyan}{\href{#1}{#2}}}

\title{北京师范大学 Beamer 模板}
\subtitle{演示文稿模板}
% \course{Master's Degree in Computer Science}
\author{作者姓名}
% \IDnumber{1234567}
\date{\today}

\begin{document}
\maketitle

\begin{frame}

本模板基于 \hrefcol{https://www.overleaf.com/latex/templates/sintef-presentation/jhbhdffczpnx}{SINTEF Presentation} 和 \hrefcol{https://github.com/TOB-KNPOB/Beamer-LaTeX-Themes}{Beamer-LaTeX-Themes} 改编

\vspace{\baselineskip}

适配北京师范大学视觉形象识别系统

\vspace{\baselineskip}

以下内容介绍如何使用 \LaTeX\ 和 Beamer 制作幻灯片

\end{frame}

\section{简介}

\begin{frame}{Beamer 简介}
\begin{itemize}
\item 我们假设你已经会使用 \LaTeX;如果不会,
\hrefcol{http://en.wikibooks.org/wiki/LaTeX/}{可以在这里学习}
\item Beamer 是 \LaTeX 中最流行、最强大的演示文稿文档类之一
\item Beamer 还有详细的
\hrefcol{http://www.ctan.org/tex-archive/macros/latex/contrib/beamer/doc/beameruserguide.pdf}{用户手册}
\item 这里我们只介绍最基本的功能,帮助你快速上手
\end{itemize}
\end{frame}

\begin{frame}{Beamer vs. PowerPoint}
与 PowerPoint 相比,使用 \LaTeX\ 更好,因为:
\begin{itemize}
\item 它不是所见即所得(WYSIWYG),而是所想即所得(WYMIWYG):\\
你编写内容,计算机负责排版
\item 生成 \texttt{pdf} 文件:没有字体、公式、程序版本的问题
\item 更容易保持一致的样式、字体、高亮等
\item \TeX\ 的数学排版是最棒的:
\begin{equation*}
\mathrm{i}\,\hslash\frac{\partial}{\partial t} \Psi(\mathbf{r},t) =
-\frac{\hslash^2}{2\,m}\nabla^2\Psi(\mathbf{r},t)
+ V(\mathbf{r})\Psi(\mathbf{r},t)
\end{equation*}

\end{itemize}
\end{frame}

\begin{frame}[fragile]{开始使用}
\framesubtitle{选择 SINTEF 主题}
要使用 \texttt{sintefbeamer},请用以下导言区开始一个 \LaTeX\ 文档:
\begin{block}{最简 SINTEF Beamer 文档}
\verb|\documentclass{beamer}|\\
\verb|\usetheme{sintef}|\\
\verb|\begin{document}|\\
\verb|\begin{frame}{Hello, world!}|\\
\verb|\end{frame}|\\
\verb|\end{document}|\\
\end{block}
\end{frame}

\begin{frame}[fragile]{标题页}
要设置典型的标题页,请在导言区调用以下命令:
\begin{block}{标题页命令}
\begin{verbatim}
\title{报告标题}
\subtitle{报告副标题}
\author{作者1、作者2}
\date{\today} % 也可以用于会议名称等
\end{verbatim}
\end{block}
然后使用 \verb|\maketitle| 命令输出标题页。

要设置 \textbf{背景图片},请在 \verb|\maketitle| 之前使用 \verb|\titlebackground| 命令;
它的唯一参数是图像文件的名称(或路径)。

如果使用 \textbf{带星号版本} \verb|\titlebackground*|,图像将被裁剪并显示在标题页右侧。

\end{frame}

\begin{frame}[fragile]{编写简单幻灯片}
\framesubtitle{非常简单!}
\begin{itemize}[<+->]
\item 典型幻灯片包含项目符号列表
\item 这些可以按顺序逐步显示
\end{itemize}
\begin{block}{项目符号列表页面代码}<+->
\begin{verbatim}
\begin{frame}{编写简单幻灯片}
  \framesubtitle{非常简单!}
  \begin{itemize}[<+->]
    \item 典型幻灯片包含项目符号列表
    \item 这些可以按顺序逐步显示
  \end{itemize}\end{frame}
\end{verbatim}
\end{block}
\end{frame}

\section{个性化设置}

\footlinecolor{sintefyellow}
\begin{frame}[fragile]{更改幻灯片样式}
\begin{itemize}
\item 你可以在导言区选择白色或 \textit{主色} \textbf{幻灯片样式},使用 \verb|\themecolor{white}|(默认)或 \verb|\themecolor{main}|
      \begin{itemize}
      \item 你\emph{不应该}在文档中更改这些:Beamer 不喜欢这样
      \item 如果你\emph{确实}需要,可能必须在幻灯片中添加
      \verb|\usebeamercolor[fg]{normal text}|
      \end{itemize}
\item 你可以使用 \verb|\footlinecolor{color}| 更改 \textbf{页脚颜色}
      \begin{itemize}
      \item 将命令放在新的 \verb|frame| \emph{之前}
      \item 有四种"官方"颜色:
      \testcolor{maincolor}、\testcolor{sintefyellow}、
      \testcolor{sintefgreen}、\testcolor{sintefdarkgreen}
      \item 默认没有页脚;可以使用 \verb|\footlinecolor{}| 恢复
      \item 其他颜色可能有效,但不保证!
      \item 不应该与 \verb|maincolor| 主题一起使用!
      \end{itemize}
\end{itemize}
\end{frame}

\begin{frame}[fragile]{文本块}
\begin{columns}
\begin{column}{0.3\textwidth}
\begin{block}{标准文本块}
颜色与页脚协调(蓝色主题中为灰色)
\begin{verbatim}
\begin{block}{标题}
内容...
\end{block}
\end{verbatim}
\end{block}
\end{column}
\begin{column}{0.7\textwidth}
\begin{colorblock}[black]{sinteflightgreen}{彩色文本块}
与左侧类似,但你可以选择颜色。默认文本为白色,但可以用可选参数设置。
\small
\begin{verbatim}
\begin{colorblock}[black]{sinteflightgreen}{标题}
内容...
\end{colorblock}
\end{verbatim}
\end{colorblock}
彩色文本块的"官方"颜色有:\testcolor{sinteflilla}、
\testcolor{maincolor}、\testcolor{sintefdarkgreen} 和
\testcolor{sintefyellow}。
\end{column}
\end{columns}
\end{frame}

\footlinecolor{}
\begin{frame}[fragile]{使用颜色}
\begin{itemize}[<alert@2>]
  \item 你可以使用 \verb|\textcolor{<颜色名称>}{文本}| 命令来使用颜色
  \item 颜色在 \texttt{sintefcolor} 包中定义:
  \begin{itemize}
  \item 主色:\testcolor{maincolor} 及其搭档 \testcolor{sintefgrey}
  \item 三种绿色:\testcolor{sinteflightgreen}、
  \testcolor{sintefgreen}、\testcolor{sintefdarkgreen}
  \item 附加颜色:\testcolor{sintefyellow}、\testcolor{sintefpurple}、
        \testcolor{sinteflilla}
        \begin{itemize}
        \item 这些可以有深浅变化——参见 \verb|sintefcolor| 文档或
        SINTEF 形象手册
        \end{itemize}
  \end{itemize}
  \item 不要\emph{滥用}颜色:\verb|\emph{}| 通常就足够了
  \item 使用 \verb|\alert{}| 来将 \alert<2->{焦点} 放在某处
  \item<2- | alert@2> 如果你高亮太多,就等于没有高亮!
\end{itemize}
\end{frame}

\begin{frame}[fragile]{添加图片}
\begin{columns}
\begin{column}{0.7\textwidth}
添加图片与普通 \LaTeX 相同:
\begin{block}{添加图片代码}
\begin{verbatim}
\usepackage{graphicx}
% ...
\includegraphics[width=\textwidth]
{assets/bnu_logo}
\end{verbatim}
\end{block}
\end{column}
\begin{column}{0.3\textwidth}
\includegraphics[width=\textwidth]
{assets/bnu_logo}
\end{column}
\end{columns}
\end{frame}

\begin{frame}[fragile]{分栏排版}
分栏排版简单且常用;
通常一边放图片,另一边放文字:
\begin{columns}
\begin{column}{0.6\textwidth}
这是第一栏
\end{column}
\begin{column}{0.3\textwidth}
这是第二栏
\end{column}
\end{columns}
\begin{block}{分栏代码}
\begin{verbatim}
\begin{columns}
    \begin{column}{0.6\textwidth}
        这是第一栏
    \end{column}
    \begin{column}{0.3\textwidth}
        这是第二栏
    \end{column}
    % 还可以有更多栏!
\end{columns}
\end{verbatim}
\end{block}
\end{frame}

\begin{chapter}[assets/bnu_background_negative]{}{特殊幻灯片}
\begin{itemize}
\item 章节幻灯片
\item 侧边图片幻灯片
\end{itemize}
\end{chapter}

\footlinecolor{maincolor}
\begin{frame}[fragile]{章节幻灯片}
\begin{itemize}
\item 类似于 \verb|frame|,但有更多选项
\item 使用 \verb|\begin{chapter}[<image>]{<color>}{<title>}| 打开
\item 图像是可选的,颜色和标题是必需的
\item 有七种"官方"颜色:\testcolor{maincolor}、
\testcolor{sintefdarkgreen}、\testcolor{sintefgreen}、
\testcolor{sinteflightgreen}、\testcolor{sintefpurple}、\testcolor{sintefyellow}、
\testcolor{sinteflilla}。
      \begin{itemize}
      \item 奇怪的是,这些比页脚的官方颜色\emph{更多}。
      \item 将后续幻灯片的页脚改为与章节幻灯片相同的颜色可能是个不错的点缀。由你选择。
      \end{itemize}
\item 除此之外,\verb|chapter| 的行为与 \verb|frame| 相同。
\end{itemize}
\end{frame}

\begin{sidepic}{assets/bnu_background_alternative}{侧边图片幻灯片}
\begin{itemize}
\item 使用 \texttt{$\backslash$begin\{sidepic\}\{<image>\}\{<title>\}} 打开
\item 除此之外,\texttt{sidepic} 的行为与 \texttt{frame} 相同
\end{itemize}
\end{sidepic}

\footlinecolor{}
\begin{frame}
\frametitle{字体}
\begin{itemize}
\item 字体的首要任务是可读性
\item 有好的字体...
  \begin{itemize}
  \item {\textrm{仅在高清晰度投影仪上使用衬线字体}}
  \item {\textsf{其他情况下使用无衬线字体(或者如果你更喜欢的话)}}
  \end{itemize}
\item ... 也有不太好的字体:
  \begin{itemize}
  \item {\texttt{永远不要用等宽字体作为正文}}
  \item {\frakfamily 哥特体、花体或奇怪的字体:应该永远避免}
\end{itemize}
\end{itemize}
\end{frame}

\begin{frame}[fragile]{外观设置}
\begin{itemize}
\item 要插入带有标题和致谢的最终幻灯片,请使用 \verb|\backmatter|。
      \begin{itemize}
      \item 标题也会与作者名一起出现在页脚中,你可以使用 \verb|\footlinepayoff| 更改此文本
      \item 你可以使用 \verb|\backmatter[notitle]| 从最终幻灯片中移除标题
      \end{itemize}
\item 宽高比默认为 16:9,你不应该为了旧投影仪而改为 4:3,
      因为将 16:9 演示文稿完美转换为 4:3 是根本不可能的;间距\emph{会}错乱
      \begin{itemize}
      \item \texttt{beamer} 类的 \texttt{aspectratio} 参数会被 SINTEF 主题覆盖
      \item 如果你\emph{确实}知道自己在做什么,可以查看包代码并寻找 \texttt{geometry} 类。
      \end{itemize}
\end{itemize}
\end{frame}

\section{总结}

\begin{frame}
\frametitle{祝你好运!}
\begin{itemize}
\item 介绍就到这里!现在你应该已经了解得够多了
\item 如果你有更正或建议,
\hrefcol{mailto:zhanghongwen@bnu.edu.cn}{请发给我!}
\end{itemize}
\end{frame}

\backmatter
\end{document}
